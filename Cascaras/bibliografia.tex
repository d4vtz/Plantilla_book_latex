%---------------------------------------------------------------------
%
%                      configBibliografia.tex
%
%---------------------------------------------------------------------
%
% Fichero  que  configura  los  par�metros  de  la  generaci�n  de  la
% bibliograf�a.  Existen dos  par�metros configurables:  los ficheros
% .bib que se utilizan y la frase c�lebre que aparece justo antes de la
% primera referencia.
%
%---------------------------------------------------------------------


%%%%%%%%%%%%%%%%%%%%%%%%%%%%%%%%%%%%%%%%%%%%%%%%%%%%%%%%%%%%%%%%%%%%%%
% Definici�n de los ficheros .bib utilizados:
% \setBibFiles{<lista ficheros sin extension, separados por comas>}
% Nota:
% Es IMPORTANTE que los ficheros est�n en la misma l�nea que
% el comando \setBibFiles. Si se desea utilizar varias l�neas,
% terminarlas con una apertura de comentario.
%%%%%%%%%%%%%%%%%%%%%%%%%%%%%%%%%%%%%%%%%%%%%%%%%%%%%%%%%%%%%%%%%%%%%%
\setBibFiles{%
nuestros,latex,otros%
}

%%%%%%%%%%%%%%%%%%%%%%%%%%%%%%%%%%%%%%%%%%%%%%%%%%%%%%%%%%%%%%%%%%%%%%
% Definici�n de la frase c�lebre para el cap�tulo de la
% bibliograf�a. Dentro normalmente se querr� hacer uso del entorno
% \begin{FraseCelebre}, que contendr� a su vez otros dos entornos,
% un \begin{Frase} y un \begin{Fuente}.
%
% Nota:
% Si no se quiere cita, se puede eliminar su definici�n (en la
% macro setCitaBibliografia{} ).
%%%%%%%%%%%%%%%%%%%%%%%%%%%%%%%%%%%%%%%%%%%%%%%%%%%%%%%%%%%%%%%%%%%%%%
\setCitaBibliografia{
\begin{FraseCelebre}
\begin{Frase}
  Y as�, del mucho leer y del poco dormir, se le sec� el celebro de
  manera que vino a perder el juicio.
\end{Frase}
\begin{Fuente}
  Miguel de Cervantes Saavedra
\end{Fuente}
\end{FraseCelebre}
}

%%
%% Creamos la bibliografia
%%
\makeBib
