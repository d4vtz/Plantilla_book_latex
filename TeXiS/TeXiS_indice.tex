%---------------------------------------------------------------------
%
%                        TeXiS_indice.tex
%
%---------------------------------------------------------------------
%
% Contiene  los  comandos  para  generar  el �ndice  de  palabras  del
% documento.
%
%---------------------------------------------------------------------
%
% NOTA IMPORTANTE: el  soporte en TeXiS para el  �ndice de palabras es
% embrionario, y  de hecho  ni siquiera se  describe en el  manual. Se
% proporciona  una infraestructura  b�sica (sin  terminar)  para ello,
% pero  no ha  sido usada  "en producci�n".  De hecho,  a pesar  de la
% existencia de  este fichero, *no* se incluye  en Tesis.tex. Consulta
% la documentaci�n en TeXiS_pream.tex para m�s informaci�n.
%
%---------------------------------------------------------------------


% Si se  va a generar  la tabla de  contenidos (el �ndice  habitual) y
% tambi�n vamos a  generar el �ndice de palabras  (ambas decisiones se
% toman en  funci�n de  la definici�n  o no de  un par  de constantes,
% puedes consultar modo.tex para m�s informaci�n), entonces metemos en
% la tabla de contenidos una  entrada para marcar la p�gina donde est�
% el �ndice de palabras.

\ifx\generatoc\undefined
\else
   \addcontentsline{toc}{chapter}{\indexname}
\fi

% Generamos el �ndice
\printindex


